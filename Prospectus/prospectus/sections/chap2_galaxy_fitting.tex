% Author: Bhishan Poudel
% Dates I worked:
%  Nov 2, 2017
%
%
%#*******************************************************
%#=======================================================
%# Chapter 2: Galaxy Fitting
%#=======================================================
%#*******************************************************
%
%
\section{Galaxy Fitting}\label{sec:chap2}
In this section, we want to fit the bulge and disk component fitting to the base 
galaxy images provided from HST survey. We use the software called 
\textit{Galfit} \footnote{https://users.obs.carnegiescience.edu/peng/work/galfit/galfit.html}
to get the reasonable components of the base galaxy. Galfit is an image analysis
algorithm that can model profiles of any given astronomical objects in the given
fits image of the data sample. For example, if an fits image contains a galaxy,
we can use galfit to fit the bulge and disk components to that galaxy.

Here, for the bulge part of the galaxy, we use the de Vaucouleurs profile in 
Galfit.
The de Vaucouleurs profile describes how the surface brightness of a giant
elliptical galaxy changes as a  function of the radius R from the center of the
galaxy.

Let $R_e$ be the radius of an isophote containing the half of the total 
luminosity for a galaxy, then, for a de Voucouleurs profile, the surface 
brightness enclosed by the radius $R$ in that galaxy is given by:

\begin{eqnarray}\label{[eq:devauc]}
    I(R) = I_e e^{-7.669 [ (R/R_e)^{1/4} -1]  }
\end{eqnarray}

Galfit has this formula as a built-in feature to get the bulge component of the
galaxy.

Similarly, to fit the disk profile to a galaxy, we use the profile called 
\textit{exponential disk profile} in Galfit. This exponential disk profile 
is a special case of Sersic profile. In the Sersic
profile, the total surface brightness enclosed by the radius R around the center
of a galaxy is given by

\begin{eqnarray}\label{[eq:sersic]}
    ln \ I(R) = ln \ I_0 - k R^{1/n}
\end{eqnarray}

where, $I_0$ is the central surface brightness at $R = 0$ and the parameter n is 
called \textit{Sersic index} which determines the curviness of the Sersic profile.

For the most of the spiral galaxies and dwarf elliptical galaxies, the Sersic index
is close to 1. This case when the Sersic index n is equal to 1 is called
exponential disk profile.
In exponential disk profile, the surface brightness inside the radius R is given
by

\begin{eqnarray}\label{[eq:expdisk]}
    ln \ I(R) = ln \ I_0 - k R
\end{eqnarray}

In this project we have 201 base galaxies obtained from the HST survey. As we 
know that some of the galaxies have both bulge and disk components and some
do not have it. For the case, when the base galaxy has bulge-disk components,
Galfit gives nice bulge and disk components, however, when the base galaxy itself
does not have reasonable bulge-disk component, Galfit can not give bulge and disk
components of that galaxy. In that case either Galfit fails to give two 
components or, gives bad parameter. We can see the whether the fitted parameters
are good or bad in the log file created from the Galfit. If a parameter is 
enclosed by \* then, we can not trust the fitting, and we treat like the given
galaxy does not have reasonable bulge-disk components.

Let's say we get two successful bulge and disk components of
a base galaxy, then we choose devauc profile as the bulge and exponential profile
as the disk image. On the other hand, if either the Galfit fails to produce
two component fittings or gives non-reliable fitting parameters, then we make some 
assumptions how to choose bugle and disk components. 
If the single component fitting of \textit{devauc profile} gives the better fit 
parameters than single component fitting of \textit{expdisk profile} ,
we choose the base galaxy as the bulge component and we
choose an empty image as the disk component. 
Similarly, if the single component \textit{expdisk profile} gives better fitting than 
single component \textit{devauc} fitting, we choose the base galaxy as the disk component 
and we choose an empty image as the bulge component. We may notice that,
while doing the galaxy fitting to our 201 base galaxies, we found that most the galaxies
gave better fit for the \textit{devauc profile} than the \textit{expdisk profile}.
An example of input parameter file for \textit{Galfit} is shown next page.
\newpage

\verbatiminput{sections/expdisk_devauc.sh}


Another point to note is that, to use the Galfit program, we need to use a PSF image
to convolve the original galaxy with the given PSF. The PSF image is different for 
different filter images of a given galaxy. That is, for F814W filter of a base
galaxy $galaxy\_f814w0.fits$ we use the psf $psf\_f814w.fits$ and for the galaxy
$galaxy\_f660w0.fits$ we use the psf $psf\_f606w.fits$. For all the F814W filter
images we use the same $psf\_f814w.fits$ and for all the F606W filter images we use
the same $psf\_f066w.fits$.

Here, in our project we use the F814W filter images taken from the HST ACS Wide Field Channel Camera.
To create the relevant PSF, we use an on-line tool called  \textit{STScI TinyTim Web Application} 
\footnote{http://www.stsci.edu/hst/observatory/focus/TinyTim}.

The TinyTim web application needs some parameters to create a psf.
For this project we chose the following parameters:

\begin{table}[!h]
\centering
\caption{Tiny Tim Parameters}
\label{tiny-tim}
\begin{tabular}{ll}
Camera         & ACS - Wide Field Channel \\
Chip           & 1                        \\
Pixel Position & 301 301                  \\
Filter         & F814w                    \\
Spectrumtype   & Blackbody                \\
Spectrumvalue  & 6000                     \\
PSF diameter   & 5.0 arcsec               \\
Focus          & 0.0                     
\end{tabular}
\end{table}

